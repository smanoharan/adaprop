\documentclass[a4paper,12pt]{report} % use larger type; default would be 10pt

\usepackage{amsmath}
\usepackage{amssymb}
\usepackage[parfill]{parskip} % Begin paragraphs with an empty line rather than an indent
\usepackage{apacite} % apa style citations and references
\usepackage{fullpage} % reduce margins
\usepackage{booktabs}
\usepackage{multirow} 
\usepackage{paralist} % very flexible & customisable lists (eg. enumerate/itemize, etc.)
\usepackage{url} % for rendering urls (in bibliography)
\usepackage{xspace} % for spaces after macros
\usepackage{setspace}
\usepackage{algorithm}
\usepackage{algpseudocode}
\usepackage{comment}

% For figures and subfigures
\usepackage{graphicx}
\usepackage{caption}
\usepackage{subcaption}

\usepackage{pgfplots} % For plots
\pgfplotsset{compat=1.5}
\usepackage[section]{placeins} % avoid figures leaving sections

\usepackage{tikz} % For graphs 
\usetikzlibrary{positioning}

%% ABBREVIATIONS
\newcommand{\AdaProp}{\texttt{AdaProp}\xspace}
\newcommand{\real}{\mathbb{R}}
\newcommand{\mcl}[1]{\mathcal{#1}}
\newcommand{\power}{\mathbb{P}}
\newcommand{\vect}[1]{\boldsymbol{#1}}

\DeclareMathOperator{\avg}{avg}
\DeclareMathOperator{\tsum}{sum}

%\linespread{1.3} % 1.3 *is* one-and-a-half spacing

\title{\ \\ \  \\ Adaptive Propositionalisation of Multi-Instance Data Towards Image Classification}
\author{Siva Manoharan, supervised by Eibe Frank}
\date{} % No date


\begin{document}
\onehalfspacing
\pagenumbering{gobble}  % don't render a page number on the first page

\maketitle 
%\ \\ \ \\ \ \\ \ \\ \ \\ \ \\ \ % Some blank lines
\begin{abstract}
Multi-instance learning is a type of supervised machine learning 
    where the class labels are attached to bags of instances. 
This is in contrast to single-instance learning 
    where each individual instance is given a class label.
Propositionalisation is the process of 
    converting a multi-instance dataset into a single-instance dataset, 
    allowing standard machine learning algorithms to learn the converted dataset. 
In this project, 
    we examine an adaptive approach to propositionalising multi-instance data, 
    where the single-instance base learner is 
    used to guide the propositionalisation process, 
    resulting in a converted dataset 
    which is more suited to the specific base learner.
In the future, 
    we will also examine the application of multi-instance learning 
    to image classification and 
    evaluate the algorithm developed in this project 
    on existing image classification datasets.
\end{abstract}

\thispagestyle{empty}
\clearpage
\pagenumbering{arabic}  

\chapter{Introduction} 

Multi-instance (MI) learning is 
    machine learning over multi-instance data, 
    where the instances are grouped together into bags and 
    the learning is performed over the bags 
    rather than the individual instances.
Multi-instance learning was originally proposed 
    in the context of drug-activity prediction.
One approach to handling multi-instance data is propositionalisation, 
    where each bag of instances is converted into 
    a single feature vector. 
This converted dataset can be used with 
    standard single-instance machine learning algorithms 
    such as SVMs and neural networks.
This project explores adaptive propositionalisation, 
    where the single-instance base learner is used to 
    make decisions when propositionalising the dataset. 
The propositionalisation approach considered is the 
    partitioning of the instance space into regions, 
    where each region corresponds to 
    one or more features in the propositionalised dataset.

An application of multi-instance learning is image classification,
    where, for example, 
    an image is represented as a bag of segments and 
    the interaction between the segments contributes 
    to the class of the image. 
In this context, 
    each segment is an instance and 
    thus each image is a bag of instances. 
In the future, we will examine this application of multi-instance learning.

\tableofcontents

\chapter{Background}

[TODO]
\section{Multi Instance Learning}
\section{Propositionalisation}
\section{Application to Image Classification}

Multi-instance learning was introduced by \citeA{Diet97} 
    in the context of drug activity prediction. 
The multi-instance learning problem was 
    defined as a two-class supervised learning problem where 
    the classification occurs over bags of instances 
    rather than the individual instances. 
\citeA{Diet97} assumed that each instance had a hidden class label and 
    that a bag belongs to the positive class if and only if 
    at least one of the instances in the bag belongs to the positive class. 
This assumption is known as the ``standard MI assumption'' and 
    fits the original domain of drug activity prediction well.

Multi-instance learning has also been applied to other domains 
    such as image classification and text classification where 
    the standard MI assumption is less appropriate. 
Therefore, a number of generalisations to the multi-instance problem have been proposed.
\citeA{Chen2006} proposed a generalised MI assumption in which 
    the label of each bag is determined by the 
    distance of the bag's instances to some hidden target points in instance space.
Similarly, \citeA{Weidmann2003} proposed a hierarchy of MI assumptions where 
    the bag's class is determined by the number of instances 
    of the bag which are located in certain regions of instance space.
This project uses the same generalised MI assumptions as \citeA{Weidmann2003}.

There have been a number of proposed algorithms 
    for dealing with multi-instance data. 
The original algorithm proposed by \citeA{Diet97} 
    used axis parallel rectangles to identify the regions of the instance space 
    where the positive instances lay. 
The algorithm was designed to find the regions which 
    contained at least one instance from each positive bag 
    but no instances from the negative class. 
\citeA{Maron98mil} proposed the Diverse Density algorithm, 
    where the aim was to find target points in instance space 
    which are close to at least one instance from each positive bag and 
    far away from all instances in negative bags.

\citeA{Weidmann2003} proposed a two level approach for dealing with multi-instance data 
    subject to the generalised MI assumptions. 
In the first level, 
    a decision tree was built to partition the instance space. 
In the second level, 
    the occupancy counts of instances of each bag in each region 
    was used to build a propositionalised dataset, 
    to which a single instance learner was applied.
The algorithm being developed in this project uses a similar process 
    as \citeA{Weidmann2003} in the second level, 
    but we consider an adaptive approach to partition the instance space.

The problem of image classification fits the multi-instance learning setting well. 
Since an image may contain multiple objects, 
    a single feature vector may not capture all 
    of the information in the image. 
As a MI learning problem, 
    an image can be considered to be a bag of instances, 
    where each instance is a segment of interest in the image. 
There is existing work on applying MI learning to image classification: 
    \citeA{Maron98scene} performed learning over scene images by 
        using fixed size partitions of the image. 
    \citeA{Zhang2002} used K-means to segment the image. 
In the future, 
    we will optimise the algorithm developed in this project to perform well 
    with MI image classification datasets.
%[http://www.cs.cmu.edu/~juny/MILL/review.htm] 

\chapter{Method}

Our approach, named \AdaProp, is an adaptive propositionalisation algorithm 
    for multi-instance datasets.
\AdaProp divides the single-instance space
    into regions and then propositionalises each bag 
    by computing summary statistics for the 
    subset of instances of the bag which lie in each region.
\AdaProp determines the regions by repeatedly splitting the instance space into 
    two partitions.
Therefore, \AdaProp consists of three major components: 
    a base learner (any single-instance learning algorithm),
    a process for partitioning the instance space and 
    a process for constructing the propositionalised dataset using the regions.

\section{Definitions}
A multi-instance dataset $\mcl{D}$ is a set of labelled bags, 
    where each labelled bag is a set of instances with a class label.
Each instance in each bag has a set of $k$ attributes
    thus can be considered to be a vector in $\real^k$
    (assuming all attributes are numeric).
Therefore, the set of all instances, $\mcl{I}$,
    can be defined as a set of $k$-dimensional vectors.
Thus, $ \mcl{I} \subseteq \real^k $.

Each labelled bag in the dataset $\mcl{D}$ is 
    composed of a set of instances from $\mcl{I}$,
    along with a class label. % FOR SIMPLICITY, we ignore duplicate instances
Let $\mcl{C}$ be the set of all possible class labels.
Then we can define $\mcl{D}$,
    the multi-instance dataset, as 
    $\mcl{D} \subseteq ( \power (\mcl{I}) \times \mcl{C} )$,
    where $\times$ denotes the cartesian product.

The propositionalisation process can be viewed as a function
    mapping each labelled bag $({B_i},c_i) \in \mcl{D}$ to 
    a single labelled instance $\vect{p_i} \in ( \real^j \times \mcl{C} )$.
Note that $\vect{p_i}$ has $j$ (non-class) attributes, 
    which need not be equal to $k$, the number of attributes of 
    each instance in the original dataset.
Also note that the propositionalisation process 
    does not modify the class label.
       

\section{Partitioning the instance space}

For the purposes of finding the best instance-space partitioning hyperplane,
    we first compute $\mcl{M}$, an intermediate labelled dataset 
    consisting of all instances in $\mcl{D}$.
$\mcl{M}$ is built up by collecting together all instances
    from all the bags in the dataset and 
    attaching the class label of each bag to 
    each instance in the bag.
Formally, $\mcl{M} \subseteq (\mcl{I} \times \mcl{C})$, 
    where each instance of $\mcl{M}$ appears in $\mcl{D}$:
    $$
        \forall~(\vect{a},c_i)\in\mcl{M} ~:~ 
            \exists~({B},c_b) \in \mcl{D} ~:~ 
                \vect{a} \in {B} ~\land~ c_i = c_b
    $$
    and each instance of $\mcl{D}$ appears in $\mcl{M}$:
    $$
        \forall~({B},c_b) \in \mcl{D} ~:~ 
            \forall~\vect{a} \in {B} ~:~
                (\vect{a},c_b) \in \mcl{M}
    $$    

For example, consider the small two-class dataset in Table~\ref{tEgData}, 
    consisting of two bags ($|\mcl{D}| = 2$)
    containing five instances in total, each with two attributes ($k=2$).
Figure~\ref{visM} shows the intermediate labelled dataset $\mcl{M}$ 
    for this example dataset with 
    positive instances rendered as squares and 
    negative instances rendered as triangles.

\begin{table}
\begin{center}
\begin{tabular}{ccccc}
    \toprule
    Bag & Class & Instance & Attribute $a_1$ & Attribute $a_2$ \\
    \midrule
    \multirow{2}{*}{$b_1$} & \multirow{2}{*}{positive} & $i_1$ & 0.3 & 0.7 \\
    &  & $i_2$ & 0.5 & 0.1 \\
    \cmidrule(r){1-5}
    \multirow{3}{*}{$b_2$} & \multirow{3}{*}{negative} & $i_3$ & 0.2 & 0.9 \\
    & & $i_4$ & 0.8 & 0.6 \\
    & & $i_5$ & 0.5 & 0.7 \\
    \bottomrule
    
\end{tabular}
\end{center}
\caption{Example dataset}
\label{tEgData}
\end{table}

\begin{figure}
\begin{center}
\begin{tikzpicture}
\begin{axis}[xlabel={$a_1$}, ylabel={$a_2$},
 xmin=0,xmax=1,ymin=0,ymax=1,
 scatter/classes={p={mark=square*,blue},n={mark=triangle*,red}}]
\addplot[scatter,only marks,scatter src=explicit symbolic]
table[meta=label] {
	x	y	label
	0.3	0.7	p
	0.5	0.1	p
	0.2	0.9	n
	0.8	0.6	n
	0.5	0.7	n
};
\end{axis}
\end{tikzpicture}
\end{center}
\caption{Visualisation of $\mcl{M}$ for the example dataset}
\label{visM}
\end{figure}

After computing $\mcl{M}$, 
    our algorithm
    aims to partition the instance space of $\mcl{M}$ into regions.
These regions are found by an iterative greedy algorithm (Algorithm~\ref{algoTree}), 
    which builds up a tree of partitioning hyperplanes.
For the purposes of this algorithm, 
    a partitioning hyperplane is a hyperplane in the instance space, 
    of the form $\vect{w} \cdot \vect{a} = c$,
    where $c$ and $\vect{w}$ are parameters of the hyperplane and 
    $\vect{a}$ is the vector of attribute values.
This hyperplane represents a natural partitioning of the instance space
    into two regions:
        the instances which lie above the hyperplane, 
            i.e.\ $\vect{i} \in \mcl{I}$ where $\vect{w} \cdot \vect{i} > c$; and
        the instances which lie at or below the hyperplane,
            i.e.\ $\vect{i} \in \mcl{I}$ where $\vect{w} \cdot \vect{i} \leq c$.
    
At each iteration,
    the algorithm generates a list of candidate partitioning hyperplanes and
    selects the best hyperplane to add to the tree.
To reduce the search space,
    \AdaProp, as evaluated in this project,
     only considers hyperplanes which
    correspond to testing just one attribute, 
    i.e.\ hyperplanes which intersect exactly one axis.
Therefore, each partition in this algorithm
    corresponds to a hyperplane of the form $a_i = c$.

\begin{algorithm}
\caption{Building a tree of partitioning hyperplanes}
\label{algoTree} 
\begin{algorithmic}
\State Initialise $T$, the tree of partitioning hyperplanes as a tree with a single node
\While{$T$ is not satisfactory} \hfill (Section~\ref{secStopCond})
    \State $n_i ~~\gets$ Select a leaf node in the tree $T$ \hfill (Section~\ref{secTreeBuild})
    \State $\mcl{M}_i \gets \{ (\vect{a},c) \in \mcl{M} ~|~ \vect{a}$ 
        lies in the region corresponding to $n_i \}$
    \State $B ~~\gets$ Generate candidate partitioning hyperplanes for 
        $\mcl{M}_i$ \hfill (Section~\ref{secCandGen})
    \State $B_i \,~\gets$ Select the optimal hyperplane in $B$ \hfill (Section~\ref{secOptPartition})
    \State Make $n_i$ an internal node in $T$, corresponding 
        to the hyperplane $B_i$
\EndWhile
\end{algorithmic}
\end{algorithm}

\begin{figure}
\begin{center}
\begin{tikzpicture}
\begin{axis}[xlabel={$a_1$}, ylabel={$a_2$},
 xmin=0,xmax=1,ymin=0,ymax=1,
 scatter/classes={p={mark=square*,blue},n={mark=triangle*,red}}]
\addplot[scatter,only marks,scatter src=explicit symbolic]
table[meta=label] {
	x	y	label
	0.3	0.7	p
	0.5	0.1	p
	0.2	0.9	n
	0.8	0.6	n
	0.5	0.7	n
};
\addplot[scatter src=explicit symbolic]
table {
	x	y
	0.4	0.0
	0.4	1.0
};
\addplot[scatter src=explicit symbolic]
table {
	x	y
	0.0 0.8
	0.4 0.8
	0.4	0.5
	1.0	0.5
};
\end{axis}
\end{tikzpicture}
\end{center}
\caption{Partitioning of $\mcl{M}$ for the example dataset}
\label{visMpart}
\end{figure}

\begin{figure}
\begin{center}
\begin{tikzpicture}[every node/.style={inner sep=1pt}]
\node[label=above:$n_0$] (n)   {$\bullet$};
\node[label=left:$n_1$]  (n0)  [below left=  3cm and 3cm of n]  {$\bullet$};
\node[label=right:$n_2$] (n1)  [below right= 3cm and 3cm of n]  {$\bullet$};
\node[label=left:$n_3$]  (n00) [below left=  3cm and 1cm of n0] {$\bullet$};
\node[label=right:$n_4$] (n01) [below right= 3cm and 1cm of n0] {$\bullet$};
\node[label=left:$n_5$]  (n10) [below left=  3cm and 1cm of n1] {$\bullet$};
\node[label=right:$n_6$] (n11) [below right= 3cm and 1cm of n1] {$\bullet$};

\path[thick,-] (n)  edge node[label=left :{$a_1 \leq 0.4~~$}] {} (n0);
\path[thick,-] (n)  edge node[label=right:{~~$a_1 > 0.4$}] {} (n1);
\path[thick,-] (n0) edge node[label=left :{$a_2 \leq 0.8~~$}] {} (n00);
\path[thick,-] (n0) edge node[label=right:{~~$a_2> 0.8$}] {} (n01);
\path[thick,-] (n1) edge node[label=left :{$a_2 \leq 0.5~~$}] {} (n10);
\path[thick,-] (n1) edge node[label=right:{~~$a_2> 0.5$}] {} (n11);
\end{tikzpicture}
\end{center}
\caption{A tree corresponding to the partitioning of $\mcl{M}$ in Figure~\ref{visMpart}}
\label{visMTree}
\end{figure}
    
Figure~\ref{visMpart} shows a possible partitioning of 
    the instance space for the example dataset (from Table~\ref{tEgData}).
Figure~\ref{visMTree} shows the corresponding tree of partitioning hyperplanes built by
    the algorithm.

\pagebreak

\subsection{Stopping Conditions}
\label{secStopCond}

The algorithm iterates while none of the following stopping conditions are met:
\begin{description}

\item[~~~The tree becomes too big:] \ \\
    $Depth(T) \geq maxDepth$, 
    where  $maxDepth$ is a user specified parameter.
\item[~~~Each region corresponding to a leaf node contains too few instances:] \ \\ 
    $\forall~n_i \in LeafNodes(T) ~:~ Occupancy(n_i) < minOcc$, 
    where $Occupancy(n_i)$ is the number of instances of $\mcl{M}$ which lie in $n_i$ and
    $minOcc$ is a user specified parameter.
\item[~~~The tree is sufficiently accurate for the training set:] \ \\
    $Error(BaseLearner, \mcl{D}_T) < minErr$, 
    where $\mcl{D}_T$ is the dataset $\mcl{D}$ propositionalised using $T$ (see Section~\ref{secProp}),
    $Error$ is the misclassfication error on the training set when 
    the $BaseLearner$ is trained on $\mcl{D}_T$,
    and $minErr$ is a user specified parameter.

\end{description}

\subsection{Selecting a leaf node to expand}
\label{secTreeBuild}

At each iteration of the algorithm, 
    a leaf node of $T$ is selected to be expanded 
    (into an internal node of the tree).
The leaf nodes which are eligible to be expanded
    are those which, when expanded, will not 
    violate the $maxDepth$ condition.
Let $\mcl{N}$ be the subset of the leaf nodes of $T$ 
    which are eligible for expansion:
\begin{algorithmic}
    \State $\mcl{N} \gets \{ n_i \in LeafNodes(T) ~\big|~ Depth(n_i) < maxDepth - 1 \} $
\end{algorithmic}
We assume that the nodes in $T$ are indexed in breadth first order, 
    as shown in Figure~\ref{visMTree}.
Then, among any subset of nodes with equal depth,
    the leftmost node is the node which has the least index.
This concept of leftmost-node is used to break ties in the 
    leaf node selection strategies.    
\AdaProp supports three leaf node selection strategies:

\begin{description}
\item[~~~Depth first search:] \ \\
    The node selected is the leaf node 
    which is leftmost node among 
    the leaf nodes which have the greatest depth
    in $\mcl{N}$.
\begin{algorithmic}
    \State $ greatestDepth \gets \max\{ depth(n_j) ~\big|~ n_j \in \mcl{N} \} $
    \State $ n_i \gets \textrm{leftmost}\{ n_j \in \mcl{N} ~\big|~ depth(n_j) = greatestDepth \} $
\end{algorithmic}
        
\item[~~~Breadth first search:] \ \\
    The node selected is the leaf node 
    which is leftmost node among 
    the leaf nodes which have the least depth
    in $\mcl{N}$.
\begin{algorithmic}
    \State $ leastDepth \gets \min\{ depth(n_j) ~\big|~ n_j \in \mcl{N} \} $
    \State $ n_i \gets \textrm{leftmost}\{ n_j \in \mcl{N} ~\big|~ depth(n_j) = leastDepth \} $
\end{algorithmic}
    
\item[~~~Best first search:] \ \\
    Given a heuristic function $h(n)$ such that
    $h(n_j) < h(n_i)$ iff $n_j$ is a better node to expand than $n_i$,
    the node selected is the leaf node 
    which is the leftmost among 
    the leaf nodes which have the least value for $h(n)$
    in $\mcl{N}$.
\begin{algorithmic}
    \State $ leastH \gets \min\{ h(n_j) ~\big|~ n_j \in \mcl{N} \} $
    \State $ n_i \gets \textrm{leftmost}\{ n_j \in \mcl{N} ~\big|~ h(n_j) = leastH \} $
\end{algorithmic}

    \AdaProp supports only one heuristic function, 
        the $Error(n)$, which is the training set error 
        when the base learner is trained on the dataset $\mcl{D}$
        propositionalised using $T$ along with 
        the best partitioning hyperplane at $n$.
     In order to compute the value of this heuristic function at a node $n$,
         all possible bi-partitions at $n$ must be evaluated,
         thus $Error(n)$ is computationally expensive.
    
\end{description}

\subsection{Generating Candidate partitioning hyperplanes}
\label{secCandGen}

Given the set of labelled instances $\mcl{M}_i \subseteq \mcl{M}$,
    we wish to generate a set of candidate partitioning hyperplanes for $\mcl{M}_i$,
    each of which is represented by a hyperplane in the instance space.
As mentioned above, 
    \AdaProp only considers hyperplanes 
    which intersect exactly one axis
    (thus is parallel to all other axes).
Furthermore, we attempt to find balanced partitioning hyperplanes, 
    i.e.\ those which lie somewhat near the center of the instances
    in $\mcl{M}_i$.

Each method of generating candidate partitioning hyperplanes 
    considers each attribute of the instances and attempts 
    to find midpoints of the instances along the attribute.
Therefore, most methods follow the same template:
    \begin{algorithmic}
    \For{$j = 1 \to k$} 
        \State $V_j \gets \{ a_j ~\big|~ \exists c \in \mcl{C} ~:~ (\vect{a},c) \in \mcl{M}_i \}$
        \State $m \gets $ Find $midpt(V_j)$
        \State Output candidate hyperplane: $a_j = m $
    \EndFor
    \end{algorithmic}
        
\AdaProp supports four methods for generating candidate partitioning hyperplanes:
\begin{description}

\item[~~~Range based midpoint:] \ \\
    For each attribute $a_j$, 
    the range (i.e.\ minimum and maximum) of 
        values of $a_j$ 
        for the instances in $\mcl{M}_i$ is computed, 
        and a candidate hyperplane corresponding to the midpoint between the 
        minimum and maximum values is generated.
    \begin{algorithmic}
        \State $midpt(V_j) \gets \frac{\min(V_j) ~+~ \max(V_j)}{2}$
    \end{algorithmic}
    
\item[~~~Mean:]  \ \\
    For each attribute $a_j$, 
        the candidate hyperplane generated corresponds to 
        the mean value of $a_j$ 
        for all instances in $\mcl{M}_i$.
    \begin{algorithmic}
        \State $midpt(V_j) \gets \textrm{mean}(V_j)$
    \end{algorithmic}    

\item[~~~Median:]  \ \\
    For each attribute $a_j$, 
        the candidate hyperplane generated corresponds to 
        the median value of $a_j$ 
        for all instances in $\mcl{M}_i$.
    \begin{algorithmic}
        \State $midpt(V_j) \gets \textrm{median}(V_j)$
    \end{algorithmic}  
    
\item[~~~Class boundaries:]  \ \\
    This method is different to the above three methods
        as it (potentially) generates multiple candidate partitions 
        per attribute.
    For each attribute $a_j$, 
        the values of $a_j$ for all instances in $\mcl{M}_i$ are sorted,
        and the values at which the class changes (i.e.\ the class boundaries)
        are used to generate the partitioning hyperplanes.
    This is similar to the discretization process of OneR~\cite{holte}.
    \begin{algorithmic}
    \For{$j = 1 \to k$} 
        \State $W_j \gets \{ (a_j,c) ~\big|~ (\vect{a},c) \in \mcl{M}_i \}$
        \State Sort $W_j$
        \ForAll{values $(w_j,c_j) \in W_j$ where $c_{j-1} \neq c_j$}
            \State $m \gets \frac{w_{j-1}+w_j}{2}$
            \State Output candidate hyperplane: $a_j = m$
        \EndFor
    \EndFor
    \end{algorithmic}
\end{description}

See Figure~\ref{visCandGen} for a visualisation of all the 
    candidate partitioning hyperplanes generated by each method
    for the example dataset in Table~\ref{tEgData}.

\FloatBarrier

%% Visualisation of all candidate partitioning hyperplanes:
\begin{figure}
\begin{center}
        \begin{subfigure}[b]{0.45\textwidth}
                \centering
                \begin{tikzpicture}[scale=0.8]
					\begin{axis}[xlabel={$a_1$}, ylabel={$a_2$},
					 xmin=0,xmax=1,ymin=0,ymax=1,
					 scatter/classes={p={mark=square*,blue},n={mark=triangle*,red}}]
					\addplot[scatter,only marks,scatter src=explicit symbolic]
					table[meta=label] {
						x	y	label
						0.3	0.7	p
						0.5	0.1	p
						0.2	0.9	n
						0.8	0.6	n
						0.5	0.7	n
					};
					\addplot[scatter src=explicit symbolic,dashed]
					table {
						x	y
						0.5	0.0
						0.5	1.0
					};
					\addplot[scatter src=explicit symbolic,dashed]
					table {
						x	y
						0.0 0.5
						1.0 0.5
					};
					\end{axis}
			    \end{tikzpicture}
                \caption{Range based midpoint}
        \end{subfigure}%
        ~~~
        \begin{subfigure}[b]{0.45\textwidth}
                \centering
                \begin{tikzpicture}[scale=0.8]
					\begin{axis}[xlabel={$a_1$}, ylabel={$a_2$},
					 xmin=0,xmax=1,ymin=0,ymax=1,
					 scatter/classes={p={mark=square*,blue},n={mark=triangle*,red}}]
					\addplot[scatter,only marks,scatter src=explicit symbolic]
					table[meta=label] {
						x	y	label
						0.3	0.7	p
						0.5	0.1	p
						0.2	0.9	n
						0.8	0.6	n
						0.5	0.7	n
					};
					\addplot[scatter src=explicit symbolic,dashed]
					table {
						x	y
						0.46	0.0
						0.46	1.0
					};
					\addplot[scatter src=explicit symbolic,dashed]
					table {
						x	y
						0.0 0.6
						1.0 0.6
					};
					\end{axis}
			    \end{tikzpicture}
                \caption{Mean}
        \end{subfigure}%
        \\ \  \\
        \begin{subfigure}[b]{0.45\textwidth}
                \centering
                \begin{tikzpicture}[scale=0.8]
					\begin{axis}[xlabel={$a_1$}, ylabel={$a_2$},
					 xmin=0,xmax=1,ymin=0,ymax=1,
					 scatter/classes={p={mark=square*,blue},n={mark=triangle*,red}}]
					\addplot[scatter,only marks,scatter src=explicit symbolic]
					table[meta=label] {
						x	y	label
						0.3	0.7	p
						0.5	0.1	p
						0.2	0.9	n
						0.8	0.6	n
						0.5	0.7	n
					};
					\addplot[scatter src=explicit symbolic,dashed]
					table {
						x	y
						0.5	0.0
						0.5	1.0
					};
					\addplot[scatter src=explicit symbolic,dashed]
					table {
						x	y
						0.0 0.7
						1.0 0.7
					};
					\end{axis}
			    \end{tikzpicture}
                \caption{Median}
        \end{subfigure}%
        ~~~
        \begin{subfigure}[b]{0.45\textwidth}
                \centering
                \begin{tikzpicture}[scale=0.8]
					\begin{axis}[xlabel={$a_1$}, ylabel={$a_2$},
					 xmin=0,xmax=1,ymin=0,ymax=1,
					 scatter/classes={p={mark=square*,blue},n={mark=triangle*,red}}]
					\addplot[scatter,only marks,scatter src=explicit symbolic]
					table[meta=label] {
						x	y	label
						0.3	0.7	p
						0.5	0.1	p
						0.2	0.9	n
						0.8	0.6	n
						0.5	0.7	n
					};
					\addplot[scatter src=explicit symbolic,dashed]
					table {
						x	y
						0.25	0.0
						0.25	1.0
					};
					\addplot[scatter src=explicit symbolic,dashed]
					table {
						x	y
						0.5	0.0
						0.5	1.0
					};
					\addplot[scatter src=explicit symbolic,dashed]
					table {
						x	y
						0.0 0.8
						1.0 0.8
					};
					\addplot[scatter src=explicit symbolic,dashed]
					table {
						x	y
						0.0 0.7
						1.0 0.7
					};
					\end{axis}
			    \end{tikzpicture}
                \caption{Class boundaries}
        \end{subfigure}
\end{center}
\caption{Candidate partitions generated for the $\mcl{M}$ corresponding to the example dataset}
\label{visCandGen}
\end{figure}

\subsection{Selecting the optimal partitioning hyperplane}
\label{secOptPartition}

Given a set of candidate partitioning hyperplanes $B$ for the labelled set of instances $\mcl{M}_i$, 
    \AdaProp aims to find the optimal hyperplane $B_i$ in $B$, 
    which is the hyperplane with the least
    training set error on $\mcl{D}$ when using the base learner.
For each candidate hyperplane $B_j$,
    a new tree $T_j$ is built by adding $B_j$ to $T$.
Then $T_j$ is used to propositionalise $\mcl{D}$
    and the base learner is trained on the propositionalised dataset.
The candidate hyperplane which results in the least training set 
    error during this process is selected as the 
    optimal hyperplane.
This process is given in Algorithm~\ref{algoOptPart}.

\begin{algorithm}
\caption{Selecting the Optimal Partitioning Hyperplane}
\label{algoOptPart} 
\begin{algorithmic}
    \Function{EvaluatePartition}{$T,n_i,B_j$}
    \State $T_j \gets T$ with the hyperplane $B_j$ added at the node $n_i$
        \State $\mcl{D}_{T_j} \gets \mcl{D}$ propositionalised using $T_j$
        \State \Return $Error(BaseLearner,\mcl{D}_{T_j})$
    \EndFunction
    \State 
    \Function{FindOptimalPartition}{$T,n_i,B$}
    \State $minErr = \min \{ \textrm{\sc EvaluatePartition}(T,n_i,B_j) ~|~ B_j \in B \}$
    \State \Return $\textrm{any} \{ B_j \in B ~|~ \textrm{\sc EvaluatePartition}(T,n_i,B_j) = minErr \}$
    \EndFunction
\end{algorithmic}
\end{algorithm}

\section{Propositionalisation}
\label{secProp}

The tree of partitioning hyperplanes $T$ defines a set of regions in the instance space.
Each such region will correspond to one or more attributes 
    in the propositionalised dataset.
\AdaProp implements two different approaches to propositionalisation of each bag
    (Algorithm~\ref{algoProp}):
    Count-based and Summary Based.


\begin{algorithm}
\caption{Propositionalisation (Given a function {\sc Prop} $: \power(\mcl{I}) \to \real^m, m \geq 1$)}
\label{algoProp} 
    \begin{algorithmic}
    \State Initialise the propositionalised dataset $\mcl{P}$ as an empty dataset
    \ForAll{bags $(B_i,c) \in \mcl{D}$}
        \State Initialise the propositionalised vector $\vect{p}$ as an empty list
        \ForAll{nodes $n_j$ in the tree $T$} 
            \State $X_j \gets \{ i \in B_i ~|~ i$ 
                lies in the region corresponding to $n_j \} $
            \State $\vect{p} \gets \vect{p} \oplus \textrm{\sc Prop}(X_j)$
            \Comment{$\oplus$ denotes concatenation}
        \EndFor
        \State Add the labelled vector $(\vect{p},c)$ 
            to the dataset $\mcl{P}$
    \EndFor
    \end{algorithmic}
\end{algorithm}

\subsection{Count-Based Propositionalisation}

In count-based propositionalisation, 
    each bag is converted into a feature vector by
    counting the number of instances of the bag which
    fall into each region.
Therefore, each region corresponds to a single attribute in the 
    propositionalised vector.
Formally, the {\sc Prop} function in Algorithm~\ref{algoProp} 
    for count-based propositionalisation is given by:
$$ \textrm{\sc Prop}: \power(\mcl{I}) \to \real,\mathrm{~where~}\textrm{\sc Prop}(X) = |X|$$
Table~\ref{tEgPropCount} shows the count-based propositionalised form of 
    the example dataset given in Table~\ref{tEgData}.

\begin{table}
\begin{center}
\begin{tabular}{*{9}{c}}
    \toprule
    Bag & $n_0$ & $n_1$ & $n_2$ & $n_3$ & $n_4$ & $n_5$ & $n_6$ & Class \\
    \midrule
    $b_1$ & 2 &     1 & 1 &     1 & 0 & 1 & 0 & positive\\
    $b_2$ & 3 &     1 & 2 &     0 & 1 & 0 & 2 & negative\\
    \bottomrule
    
\end{tabular}
\end{center}
\caption{Count-based propositionalised form of the example dataset}
\label{tEgPropCount}
\end{table}    

\subsection{Summary Based Propositionalisation}

In summary-based propositionalisation, 
    the instances which fall into each region  
    are aggregated using summary statistics, such as
    minimum, maximum, sum, and average (i.e.\ mean).
For each region, the summary statistics are computed 
    for all attributes, regardless of which attribute was 
    used in the chosen partitioning hyperplane.
Therefore, summary-based propositionalisation is more
    computationally expensive than count-based propositionalisation, 
    but is able to preserve significantly more information from each bag 
    in the propositionalisation.
Formally, the {\sc Prop} function in Algorithm~\ref{algoProp} 
    for summary-based propositionalisation is given by:
$$ \textrm{\sc Prop}:\power(\mcl{I}) \to \real^{4k}  $$
\begin{align*}
    \textrm{\sc Prop}(X) = (
        &\min(X_1),~\max(X_1),~\tsum(X_1),~\avg(X_1), \\
        &\min(X_2),~\max(X_2),~\tsum(X_2),~\avg(X_2), \\
        &\ldots, \\
        &\min(X_k),~\max(X_k),~\tsum(X_k),~\avg(X_k))
\end{align*}
Where $k$ is the number of attributes and $X_j = \{ i_j ~|~ i \in X \}$, i.e.\ 
    the value of the $j^{\mathrm th}$ attribute for all instances in $X$.
    
Table~\ref{tEgPropAgg} shows the propositionalised form of 
    the example dataset given in Table~\ref{tEgData} when
    using summary-based propositionalisation.   
    
\begin{table}
\begin{center}
\begin{tabular}{*{14}{c}}
    \toprule
        \multirow{2}{*}{Bag} & 
        \multicolumn{4}{c}{$n_0~a_1$} & 
        \multicolumn{4}{c}{$n_0~a_2$} & 
        \multicolumn{4}{c}{$n_1~a_1$} &
        \multirow{2}{*}{Class} \\
    \cmidrule(r){2-5}
    \cmidrule(r){6-9}
    \cmidrule(r){10-13}
     & min & max & sum & avg & min & max & sum & avg & min & max & sum & avg & \\
    \midrule
    $b_1$ &  0.3 & 0.5 & 0.8 & 0.4  & 0.1 & 0.7 & 0.8 & $0.40$       & 0.3 & 0.3 & 0.3 & 0.3 & positive\\
    $b_2$ &  0.2 & 0.8 & 1.5 & 0.5  & 0.6 & 0.9 & 2.2 & $0.7\dot{3}$ & 0.2 & 0.2 & 0.2 & 0.2 & negative\\
    \bottomrule    
    \multirow{2}{*}{} & 
        \multicolumn{4}{c}{$n_1~a_2$} & 
        \multicolumn{4}{c}{$n_2~a_1$} & 
        \multicolumn{4}{c}{$n_2~a_2$} &
        \multirow{2}{*}{} \\
    \cmidrule(r){2-5}
    \cmidrule(r){6-9}
    \cmidrule(r){10-13}
     & min & max & sum & avg & min & max & sum & avg & min & max & sum & avg & \\
    \midrule
    $b_1$ &  0.7 & 0.7 & 0.7 & 0.7  & 0.7 & 0.7 & 0.7 & $0.70$  & 0.1 & 0.1 & 0.1 & $0.10$ & positive\\
    $b_2$ &  0.9 & 0.9 & 0.9 & 0.9  & 0.5 & 0.8 & 1.3 & $0.65$  & 0.6 & 0.7 & 1.3 & $0.65$ & negative\\    
    \bottomrule
    
\end{tabular}
\end{center}
\caption{Summary-based propositionalised form of the example dataset 
    (showing only the nodes $n_0, n_1,$ and $n_2$).}
\label{tEgPropAgg}
\end{table}    

%\begin{comment}

\section{The base learner} 

  
Any standard machine learning algorithm can be used 
    as the base learner, i.e.\ can be used to guide the hyperplane selection process.
The choice of base learner is left upto the user, and 
    this report considers two common base learners: 
    RandomForest and LogitBoost.

RandomForest \cite{breiman} constructs an ensemble of 
    randomized decision trees, where each node of each tree is 
    built on a random subset of the attributes.
In this project, RandomForests with 100 trees were used.

LogitBoost \cite{friedman} is an adaptive boosting algorithm 
    which minimises a logisitic loss function.
In this project, LogitBoost with 30 iterations was used, 
    using Decision stumps (1-level decision trees) as the base learner.

%Algorithm(s) will be implemented using the WEKA framework.*

\section{Refinements}

The experiments (Chapter~\ref{ch:Results}) indicated that \AdaProp 
    significantly overfits some datasets.
In order to reduce this overfitting, 
    and thus improve the accuracy of \AdaProp, 
    two standard techniques were implemented: 
    cross validated parameter selection
    and randomized bagging.


\subsection{Parameter Selection}

The size of the trees of partitioning hyperplanes 
    built up by \AdaProp had substantial impact on
    the cross-validated classification accuracy.
In fact, the experiments showed that increasing the tree
    size increased the accuracy initially upto some 
    maximum, after which the accuracy declined.
This suggests that there is some optimal size 
    for the tree of partitioning hyperplanes.
From the experiments, it was also clear that this
    optimal size is dependent on the dataset, 
    i.e.\ different datasets have different optimal tree sizes.
    
In order to find the optimal tree size for any given dataset, 
    \AdaProp performs $5$-fold cross-validated selection 
    over all tree sizes (Algorithm~\ref{algoCVTSS}).
This process maintains a set of 5 trees, each built on 
    a different 80\% of the training dataset, 
    and evaluated on the remaining 20\%.
The size of each tree is then iteratively increased
    until the total error on the test datasets ceases to decrease.
This process requires more time than running 
    the \AdaProp process with a fixed tree size, 
    but provides an increase in accuracy.
However, compared to the usual method of parameter optimisation, 
    where the entire training process is repeated for each possible 
    value of the parameter, this method is significantly more efficient.

\begin{algorithm}
\caption{Cross-validated tree-size selection}
\label{algoCVTSS} 
    \begin{algorithmic}
    \State Partition, with stratification, the training set into 5 folds: 
        $\{ X_1, X_2, \ldots, X_5 \}$.
    \For{$j = 1 \to 5$}
        \State Build a tree $T_j$ of size 1, using all folds except $X_j$ as the training set.
        \State $e^0_j \gets 1$ \Comment{$e^t_j$ is the error of the tree $T_j$ when of size $t$}
        \State $e^1_j \gets$ the misclassification error of $T_j$ when evaluated on $X_j$.        
    \EndFor    
    \State $t \gets 1$ \Comment{$t$ is the current tree size}.    
    \While{$e^t_1 + e^t_2 + \ldots + e^t_5 ~\leq~ e^{t-1}_1 + e^{t-1}_2 + \ldots + e^{t-1}_5$}
        \State $t \gets t+1$
        \For{$j = 1 \to 5$}
            \State Expand $T_j$ by one node, 
                using all folds except $X_j$ as the training set.
            \State $e^t_j \gets$ the misclassification error of $T_j$ when evaluated on $X_j$.
        \EndFor
    \EndWhile
    \State Output $t-1$ as the optimal tree size.
    \end{algorithmic}
\end{algorithm}


\subsection{Randomized Bagging}

Another refinement implemented to combat overfitting is 
    randomized bagging.
Bagging was performed using WEKA's standard bagging implementation, 
    where the training set is resampled (with replacement) to generate
    $n$ independent datasets, where $n$ is the number of iterations.
Then, an ensemble of $n$ \AdaProp trees are built (i.e.\ one tree per dataset), 
    and majority voting is used to combine the predictions of the trees.

When bagging is performed on standard \AdaProp trees, 
    the runtime grows linearly with the number of iterations
    (i.e.\ the 50-iterations of bagging will take 50 times longer
    to complete than training a single \AdaProp tree).
In order to reduce the runtime and also to further reduce overfitting, 
    randomization in attributes was implemented, 
    where at each node in each tree, only a randomly chosen subset of attributes 
    was considered when selecting the optimal partitioning hyperplane.
The subset of attributes was chosen to be of size $\log_2(K)+1$, 
    where $K$ is the number of attributes, 
    therefore the randomization reduced the runtime by a factor 
    of approximately $\frac{K}{\log_2(K)}$.


\chapter{Image Classification}
...
\section{Definitions}
...
\section{Feature Selection}
...

\chapter{Results}
\label{ch:Results}


[ TODO ]
\section{Data sets}
...
\section{Hyperplane Evaluation}
...
\section{Leaf Node Selection}
...
\section{Base Learners}
...
\section{Propositionalisation}
...
\subsection{Count Based Propositionalisation}
...
\subsection{Comparison to TLC}
...
\subsection{Summary Based Propositionalisation}
...
\subsection{Comparison to RELAGGS}
...
\section{Refinements}
...
\subsection{Parameter Selection}
...
\subsection{Randomized Bagging}
...
\subsection{Comparison to Bagged TLC}
...
\subsection{Comparison to Bagged RELAGGS}
...

Experiments were conducted using 16 datasets from the UCI repository, 
    comparing the performance of three candidate hyperplane generation methods 
    (range-based midpoint, mean, and median)
    and two base learners (j48 and SMO).
All experiments were conducted for the breadth*-first search strategy for 
    building up the tree of partitioning hyperplanes.
Experiments for the the fourth candidate hyperplane generation method and best first search strategies 
    are ongoing.
    
First, we aggregate the results of all experiments and 
    compare the results for each candidate hyperplane generation method:
    
[TODO]

Then compare the results across base learners 

[ TODO ]

Then we show the results over each dataset

[ Raw Data ? can be found at appendix ? ]    
    
\chapter{Image classification Results}
\section{Feature classification}
\section{Comparison to other MI algorithms}
\section{Comparison to other image classification algorithms}

\chapter{Additional Results}
\section{Artificial dataset}
Since the results on the UCI datasets were unsatisfactory, 
    artificial datasets were generated to check that the 
    implementation of the \AdaProp algorithm.
    
Poor results with J48, overfitting the very simple dataset, better results with 
    OneR, more data and the last algorithm (what was it?)
Simpler base learner may work better in the simple datasets?

[ Learning curves ]

\section{Training Set convergence}

%\end{comment}

\chapter{Future Work}

The final output of this project is an algorithm 
    which can be used to adaptively propositionalise multi-instance data. 
Since there are other established algorithms for handling multi-instance data, 
    we will evaluate the algorithm by comparing the classification accuracy 
    on standard multi-instance datasets against methods such as 
    MILES~\cite{Chen2006}, TLC~\cite{Weidmann2003} and DD~\cite{Maron98mil}.

We will also consider the application of this algorithm towards image classification and 
    potentially convert some existing image classification
    datasets into a multi-instance representation.

Currently, \AdaProp only supports multi-instance datasets with 
    numeric attributes.
In the future, we intend to implement support for
    nominal attributes and missing values.

Since this algorithm is a meta-learner, 
    i.e.\ it is dependent on an underlying single-instance base learner, 
    a number of different base learners will be 
    used to evaluate the performance of this algorithm.
 
Finally, this algorithm will be evaluated on 
    standard image classification datasets 
    (for example the \citeA{sival}) and 
    the performance will be compared to that of 
    established image classification algorithms.

\chapter{Conclusion}
In this project, 
    we examine the propositionalisation of multi-instance data using 
    a process which is influenced by the single-instance base learner being used, 
    i.e. an adaptive propositionalisation process. 
Our approach is to partition the instance space 
    by generating a set of candidate partitions and 
    using the single instance learner to select the optimal partition. 
The result of this project is an algorithm
    which will be evaluated by comparing its performance to 
    that of existing multi-instance learning algorithms on standard MI datasets. 
In addition, 
    the application of multi-instance learning 
    to image classification will be examined and 
    the algorithm will be tuned for dealing with image classification datasets.
\clearpage

\appendix
\chapter{Result Tables}
Some data [TODO]
\chapter{Result Charts}
Some data [TODO]
\bibliographystyle{apacite} 
\bibliography{final-report}
\end{document}
