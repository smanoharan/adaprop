\documentclass{beamer}

\usepackage[english]{babel}
\usepackage{xcolor}
\usepackage{amsmath}
\usepackage{amssymb}
\usepackage{tikz}
\usepackage{pgfplots}
\usepackage{pgfplotstable}
\usetikzlibrary{plotmarks}

%\usefonttheme{serif}
%\usepackage{palatino}
\usetheme{Rochester}
\usecolortheme{beaver}
\beamertemplatenavigationsymbolsempty

\title{ ~ \\ ~ \\ {\huge AdaProp} \\ ~ \\ 
    {\large Adaptive Propositionalisation of Multi-Instance Data Towards Image Classification} \\ ~ }
\subtitle{}
\author{~ \\ ~ \\ Siva~Manoharan \\ ~ \\ Supervisor: Dr.~Eibe~Frank}
\date{}

%% Customizing colours:
\colorlet{primaryRed}{darkred!60!black} % For most Reds
\colorlet{secondaryRed}{darkred!80!black} % When Red is surrounded by grey
\colorlet{bgWhite}{white!97!black} % Near white, for page backgrounds
\colorlet{headerGrey}{white!90!black} % For Header backgrounds
\colorlet{fillGrey}{white!80!black} % For filling areas, e.g. chart areas
\colorlet{edgeGrey}{white!60!black} % For increasing apparent sharpness

\setbeamercolor*{item}{fg=primaryRed}
\setbeamercolor{title}{fg=secondaryRed} 
\setbeamercolor{background canvas}{bg=bgWhite}
\setbeamercolor{headline}{bg=headerGrey}
\setbeamertemplate{headline}[text line]{%
  \begin{beamercolorbox}[wd=\paperwidth,ht=2cm]{headline}%
  \end{beamercolorbox}%
}

%% A Macro for a column with an image and a caption
%% args: image, caption
\newcommand{\ImageAndCaptionColumn}[2]{%
    \begin{center} 
        \includegraphics[scale=0.25]{img/#1} \\~\\ 
        {\Large #2} 
    \end{center}
}

%% A Macro for frame template: a bullet point above two images, each with caption
%% args: bullet-text, img-left, caption-left, img-right, caption-right
\newcommand{\TextAndTwoImageFrame}[5]{%
    %% Default spacing is too large for just one bullet point.
    %% Reduce it:
    \vspace{-0.5cm}    
    \begin{itemize}
        \item #1
    \end{itemize}    
    \vspace{-0.5cm} % again, reducing spacing
    
    %% Two columns of image+caption pairs:
    \begin{columns}[T]
    
        \begin{column}{.5\textwidth}
            \ImageAndCaptionColumn{#2}{#3}
        \end{column}
        
        \begin{column}{.5\textwidth}
            \ImageAndCaptionColumn{#4}{#5}
        \end{column}
        
    \end{columns}
}

%% A Macro for a column and line graph for our datasets:
%% args: 
\newcommand{\ColumnAndLineGraph}[4]{%
    \begin{axis}[
        % Fixed size graph
        width=11cm, height=7cm, bar width=0.35cm,
%
        % draw only left and bottom lines
        axis x line*=bottom, axis y line*=left, draw=edgeGrey,
%
        % y-axis: 40-95, with major=10, minor=5
        ymin=40, ymax=95, minor y tick num=1,  ytick={50,60,70,80,90},
        axis y discontinuity=crunch, ylabel={Accuracy (\%)},
%
        % x-axis: category axis
        symbolic x coords={atoms,bonds,chains,musk1,musk2,trx,tiger,fox,eleph,people,bikes,cars},
        xtick=data, x tick label style={rotate=90,anchor=east}, 
%
        % Legend: at bottom right, place line first, then bar.
        reverse legend, legend pos=south east
    ]
 
        % Background bar plot
        \addplot[ybar,fill=fillGrey, draw=edgeGrey, area legend] 
            coordinates {#2};
        \addlegendentry{#1}
        
        % Foreground line plot
        \addplot+[ycomb,mark=-,draw=primaryRed,very thick,mark size=0.175cm, line legend] 
            coordinates {#4};
        \addlegendentry{#3}
    \end{axis}
}

%% uncomment for logo
%\pgfdeclareimage[height=0.5cm]{university-logo}{university-logo-filename}
%\logo{\pgfuseimage{university-logo}}

%% uncomment for stepwise appearence (of bullet points etc)
%\beamerdefaultoverlayspecification{<+->}

\begin{document}

%% remove header bar for the titlepage
{
    \makeatletter
        \setbeamercolor{headline}{bg=white!97!black}
        \def\beamer@entrycode{\vspace*{-0.8\headheight}}
    \makeatother

    \begin{frame}
        \titlepage
    \end{frame}
}

\begin{frame}{Image Classification}
    \TextAndTwoImageFrame
        {Does the Image contain a specific object?}
        {cars1}{Car}{none1}{None}
\end{frame}

\begin{frame}{Image Classification}
    \TextAndTwoImageFrame
        {Not easy in practice (for computers)}
        {cars2}{Partial Occlusion}{cars3}{Not in Foreground}
\end{frame}

\begin{frame}{Multi-Instance (MI) Learning}

    \begin{itemize}
        \item Natural for Image Classification \\ ~
        \item Text Classification and Chemical datasets \\ ~
        \item Key: Bags of Instances
    \end{itemize}
    % TODO: show splitting? (not enough time)
\end{frame}

\begin{frame}{Propositionalisation}

    \begin{itemize}
        \item Multi-instance $\to$ Single-instance \\ ~
        \item Advantage: Existing algorithms \\ ~
        \item Disadvantage: Lossy conversion
    \end{itemize}
    % TODO: Lots more algorithms in stardard mi
\end{frame}

\begin{frame}{AdaProp: Adaptive Propositionalisation}

    \begin{itemize}
        \item Use the base learner \\ ~
        \item Tree of decisions / partitions \\ ~
        \item Count by region
    \end{itemize}
    % TODO: Needs diagram

    
\end{frame}

\begin{frame}{WEKA}
  \vspace{-0.5cm}
  \ImageAndCaptionColumn{weka3}{}  
\end{frame}

\begin{frame}{Results: Impact of Parameter Selection}

    \begin{itemize}
        \item \dots \\ ~
        \item \dots \\ ~
        \item \dots
    \end{itemize}
    % TODO: needs graph
\end{frame}

\begin{frame}{AdaProp vs. Other MI algorithms}

    \begin{tikzpicture}
        \ColumnAndLineGraph{Others}{
            (atoms,83.90) (bonds,87.96) (chains,88.56)
            (musk1,90.68) (musk2,85.89) (trx,90.32)
            (tiger,84.3) (fox,67) (eleph,85.5)
            (people,82.6) (bikes,83.5)(cars,77.22)
        }{AdaProp}{
            (atoms,87.52) (bonds,89.44) (chains,89.86)
            (musk1,85.76) (musk2,82.26) (trx,88.64)
            (tiger,84.2) (fox,68.2) (eleph,84.75)
            (people,81.88) (bikes,82.5)(cars,77.08)
        }
    \end{tikzpicture}
    
\end{frame}

\begin{frame}{Summary}

  % Keep the summary *very short*.
  \begin{itemize}
  \item
      MI representation is \alert{natural} for Image Classification. \\ ~
  \item
      AdaProp propositionalises MI data using the \alert{base learner}. \\ ~
  \item
      \alert{Comparable performance} to other MI algorithms. % TODO need this?
  \end{itemize}
  
\end{frame}



\end{document}


